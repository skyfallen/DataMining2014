\documentclass{article}

\usepackage{courier}
\usepackage{graphicx} % Required for the inclusion of images
\usepackage{listings}
\usepackage{color}
\usepackage{amsmath}
\usepackage{subcaption}
\usepackage{enumitem}
\usepackage{float}
\usepackage[toc,page]{appendix}
\usepackage{dcolumn}
\usepackage{pdflscape}
\usepackage{hyperref}
\usepackage{framed}
\usepackage[english]{babel}
\usepackage{soul}
\usepackage{caption}
\usepackage{amsfonts}
\usepackage{MnSymbol,wasysym}
\usepackage{algorithmic}
\captionsetup[figure]{labelformat=empty}%

\setlength\parindent{0pt} % Removes all indentation from paragraphs

\renewcommand{\labelenumi}{\alph{enumi}.} % Make numbering in the enumerate environment by letter rather than number (e.g. section 6)

\definecolor{dkgreen}{rgb}{0,0.6,0}
\definecolor{gray}{rgb}{0.5,0.5,0.5}
\definecolor{mauve}{rgb}{0.58,0,0.82}

\lstset{frame=tb,
  language=R,
  aboveskip=3mm,
  belowskip=3mm,
  showstringspaces=false,
  columns=flexible,
  basicstyle={\small\ttfamily},
  numbers=none,
  numberstyle=\tiny\color{gray},
  keywordstyle=\color{blue},
  commentstyle=\color{dkgreen},
  stringstyle=\color{mauve},
  breaklines=true,
  breakatwhitespace=true
  tabsize=3
}

\newcommand{\footlabel}[2]{%
    \addtocounter{footnote}{1}%
    \footnotetext[\thefootnote]{%
        \addtocounter{footnote}{-1}%
        \refstepcounter{footnote}\label{#1}%
        #2%
    }%
    $^{\ref{#1}}$%
}

\newcommand{\footref}[1]{%
    $^{\ref{#1}}$%
}

%\usepackage[colorlinks]{hyperref}
\hypersetup{linkcolor=DarkRed}
\hypersetup{urlcolor=DarkBlue}
\usepackage{cleveref}

\title{Data Mining\\Homework Assignment \#11} % Title

\author{Dmytro Fishman, Anna Leontjeva and Jaak Vilo} % Author name

\begin{document}

\maketitle % Insert the title, author and date 
The goal for this homework is to get you started with basics of social network analysis. 

There are many different network analyzing tools. Choose the one you prefer. Some of the well-known tools and packages are: NetworkX and igraph for Python, JUNG for Java, igraph for R, Gephi for the visualization with some built-in calculations and even NodeXL free template for Excel.

Most of the exercises require you to check presentation slides for the definitions.

\section*{Task 1}
In this task, we will use the techniques of Social Network analysis to study a virus spread. Imagine the following situation: terrified biologists came to the Institute of Computer Science seeking for a help. Their email network was infected by the virus that was created by the student that received 'B' for his Master's Thesis and got offended. Your goal is to help poor biologists to estimate the worst-case scenario of this virus spread. 

Biologists observed that if virus infects a node, it always infects all its immediate neighbors, if they are not already infected (100\% of infection rate). Also, we know that virus travels only along the edge direction (e.g. if virus infects node A, which only has an incoming edge from node B, node B will not be infected). 

Biologists provided you with their directed anonymous email network that you can access on the course web-page. 
  
Load the data. To get the first insights about biologists' network, calculate the list of the following statistics:
\begin{itemize}
\item number of nodes in the network
\item number of edges in the network
\item number of nodes with a self-loop
\item number of mutual connections or \emph{reciprocated} edges, i.e if there is a directed edge from node a to node b, there is also an edge from b to a. 
\item number of nodes with zero indegree (those that have only outgoing edges)
\item number of nodes with zero outdegree (those that have only ingoing edges)
\item degree distribution of the given network
\item optionally calculate whatever measure you deem appropriate for better understanding
\end{itemize}
What intuition you can gather from these numbers?

\section*{Task 2}
Next, biologists ask us to estimate the vulnerability of their network. In order to measure it, you have to calculate average virus spread assuming that the initial infected node is chosen uniformly at random. What is the probability that virus will affect at least 30\% of the network nodes ('large scale' epidemics emerges).

For that, consider the following simplified version of the graph. 
\begin{figure}[H]
    \centering
    \includegraphics[width=1\textwidth]{Graph.png}
    \caption{Figure: simplified version of the graph}
    \label{fig:awesome_image}
\end{figure}

\begin{enumerate}
\item According to the virus transmission rule, calculate number of infected nodes if the initially infected was:
\begin{itemize}
\item  the node 1
\item the node 8
\item the node 12
\end{itemize} 
\item Here we introduce the notion of Bow Tie structure (\url{xx}), which is a recent concept that grasps the essence of both biological networks and the representation of links in the Internet. It divides a graph into 4 basic components (and two secondary that we omit for this task):
\begin{itemize}
\item Strongly connected component (SCC), which is the core of the graph, where all nodes can reach one other along directed links
\item  ''IN`` component that consists of nodes that can reach the SCC, but cannot be reached from it.
\item ''OUT`` component that consists of nodes that are accessible from the SCC, but do not link back to it
\item disconnected components are those nodes that are not connected to neither IN, OUT or SCC components. 
\end{itemize}
For more information on described above graph structure read article by Andrei Broder et al.: \url{http://snap.stanford.edu/class/cs224w-readings/broder00bowtie.pdf}
In this part your task is to calculate the proportion of nodes that belong to SCC, IN, OUT and disconnected components.
\item Based on estimated proportions in the part 'b', calculate the probability of emerging a 'large scale' epidemics (at least 30\% of the network nodes infected) given that the initial infected node is chosen uniformly at random from the network nodes.
\end{enumerate}
\section*{Task 3}

\section*{Task 4}
write your own function that generates erdos-renyi random graph with input parameter p, where p is the probability of edge creation. generate the graph with 50 nodes and at least five different p values. Plot the result.  
\section*{Task 5}
Community detection
\section*{Task 6}


 
\end{document}
